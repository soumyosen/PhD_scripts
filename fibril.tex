%%%%%%%%%%%%%%%%%%%%%%%%%%%%%%%%%%%%%%%%%%%%%%%%%%%%%%%%%%%%%%%%%%%%%
%% This is a (brief) model paper using the achemso class
%% The document class accepts keyval options, which should include
%% the target journal and optionally the manuscript type.
%%%%%%%%%%%%%%%%%%%%%%%%%%%%%%%%%%%%%%%%%%%%%%%%%%%%%%%%%%%%%%%%%%%%%
\documentclass[journal=jacsat,manuscript=communication]{achemso}
\usepackage{achemso}
%%%%%%%%%%%%%%%%%%%%%%%%%%%%%%%%%%%%%%%%%%%%%%%%%%%%%%%%%%%%%%%%%%%%%
%% Place any additional packages needed here.  Only include packages
%% which are essential, to avoid problems later. Do NOT use any
%% packages which require e-TeX (for example etoolbox): the e-TeX
%% extensions are not currently available on the ACS conversion
%% servers.
%%%%%%%%%%%%%%%%%%%%%%%%%%%%%%%%%%%%%%%%%%%%%%%%%%%%%%%%%%%%%%%%%%%%%
%\usepackage[version=3]{mhchem} % Formula subscripts using \ce{}
%\usepackage[T1]{fontenc}       % Use modern font encodings

%%%%%%%%%%%%%%%%%%%%%%%%%%%%%%%%%%%%%%%%%%%%%%%%%%%%%%%%%%%%%%%%%%%%%
%% If issues arise when submitting your manuscript, you may want to
%% un-comment the next line.  This provides information on the
%% version of every file you have used.
%%%%%%%%%%%%%%%%%%%%%%%%%%%%%%%%%%%%%%%%%%%%%%%%%%%%%%%%%%%%%%%%%%%%%
%%\listfiles

%%%%%%%%%%%%%%%%%%%%%%%%%%%%%%%%%%%%%%%%%%%%%%%%%%%%%%%%%%%%%%%%%%%%%
%% Place any additional macros here.  Please use \newcommand* where
%% possible, and avoid layout-changing macros (which are not used
%% when typesetting).
%%%%%%%%%%%%%%%%%%%%%%%%%%%%%%%%%%%%%%%%%%%%%%%%%%%%%%%%%%%%%%%%%%%%%
\newcommand*\mycommand[1]{\texttt{\emph{#1}}}

%%%%%%%%%%%%%%%%%%%%%%%%%%%%%%%%%%%%%%%%%%%%%%%%%%%%%%%%%%%%%%%%%%%%%
%% Meta-data block
%% ---------------
%% Each author should be given as a separate \author command.
%%
%% Corresponding authors should have an e-mail given after the author
%% name as an \email command. Phone and fax numbers can be given
%% using \phone and \fax, respectively; this information is optional.
%%
%% The affiliation of authors is given after the authors; each
%% \affiliation command applies to all preceding authors not already
%% assigned an affiliation.
%%
%% The affiliation takes an option argument for the short name.  This
%% will typically be something like "University of Somewhere".
%%
%% The \altaffiliation macro should be used for new address, etc.
%% On the other hand, \alsoaffiliation is used on a per author basis
%% when authors are associated with multiple institutions.
%%%%%%%%%%%%%%%%%%%%%%%%%%%%%%%%%%%%%%%%%%%%%%%%%%%%%%%%%%%%%%%%%%%%%
\author{Soumyo Sen}
\affiliation[University of Illinois]
{Department of Chemistry,University of Illinois, Chicago, Illinois 60607}
\author{Lela Vukovi\'c}
\affiliation[University of Texas]
{Department of Chemistry,University of Texas, El Paso, Texas 79968} 
\author{Petr Kr\'al}
\email{pkral@uic.edu}
\phone{+123 (0)123 4445556}
\fax{+123 (0)123 4445557}
\affiliation[University of Illinois]
{Department of Chemistry, Physics, Biopharmaceutical Sciences, University of Illinois, Chicago, Illinois 60607}

%%%%%%%%%%%%%%%%%%%%%%%%%%%%%%%%%%%%%%%%%%%%%%%%%%%%%%%%%%%%%%%%%%%%%
%% The document title should be given as usual. Some journals require
%% a running title from the author: this should be supplied as an
%% optional argument to \title.
%%%%%%%%%%%%%%%%%%%%%%%%%%%%%%%%%%%%%%%%%%%%%%%%%%%%%%%%%%%%%%%%%%%%%
\title[An \textsf{achemso} demo]
  {Interactions of pre-engineered nanoparticles with $A\beta40$
  fibril: energetics and structural modifications}
%   \footnote{A footnote for the title}}

%%%%%%%%%%%%%%%%%%%%%%%%%%%%%%%%%%%%%%%%%%%%%%%%%%%%%%%%%%%%%%%%%%%%%
%% Some journals require a list of abbreviations or keywords to be
%% supplied. These should be set up here, and will be printed after
%% the title and author information, if needed.
%%%%%%%%%%%%%%%%%%%%%%%%%%%%%%%%%%%%%%%%%%%%%%%%%%%%%%%%%%%%%%%%%%%%%
\abbreviations{IR,NMR,UV}
\keywords{American Chemical Society, \LaTeX}

%%%%%%%%%%%%%%%%%%%%%%%%%%%%%%%%%%%%%%%%%%%%%%%%%%%%%%%%%%%%%%%%%%%%%
%% The manuscript does not need to include \maketitle, which is
%% executed automatically.
%%%%%%%%%%%%%%%%%%%%%%%%%%%%%%%%%%%%%%%%%%%%%%%%%%%%%%%%%%%%%%%%%%%%%
\begin{document}

%%%%%%%%%%%%%%%%%%%%%%%%%%%%%%%%%%%%%%%%%%%%%%%%%%%%%%%%%%%%%%%%%%%%%
%% The "tocentry" environment can be used to create an entry for the
%% graphical table of contents. It is given here as some journals
%% require that it is printed as part of the abstract page. It will
%% be automatically moved as appropriate.
%%%%%%%%%%%%%%%%%%%%%%%%%%%%%%%%%%%%%%%%%%%%%%%%%%%%%%%%%%%%%%%%%%%%%

%%%%%%%%%%%%%%%%%%%%%%%%%%%%%%%%%%%%%%%%%%%%%%%%%%%%%%%%%%%%%%%%%%%%%
%% The abstract environment will automatically gobble the contents
%% if an abstract is not used by the target journal.
%%%%%%%%%%%%%%%%%%%%%%%%%%%%%%%%%%%%%%%%%%%%%%%%%%%%%%%%%%%%%%%%%%%%%
\begin{abstract} Self-assembly of A$\beta$40 peptides leads to an amyloid fibril
which causes Alzheimer's disease. Recently nanoparticles are used frquently in
the medicinal field. Interactions of nanoparticles with A$\beta$40 fibril are
important topic in the applications of nanotechnology in the medicinal field. We
perform atomistic molecular dynamics simulations of five different types ligated
nanoparticles with A$\beta$40 fibril. The simulations reveal that positively
charged nanoparticles can make stable interactions with fibril surface.  The
modes and the strength of the interactions can be controlled by changing the
nature of the ligands. The binding of the nanoparticles on the fibril surface
affect the structure of the fibril.  It changes the twist angle of the peptide
and also changes the number of heavy atom contacts significantly in the fibril.
All the nanoparticles affect the overall potential energy of $\beta$ sheet of
the fibril depending on the nature of ligands.  We also investigate the binding
of the nanoparticles at the tip of the fibril.  The result shows that only
negatively charged particles can make stable interactions at this
position.\end{abstract}

%%%%%%%%%%%%%%%%%%%%%%%%%%%%%%%%%%%%%%%%%%%%%%%%%%%%%%%%%%%%%%%%%%%%%
%% Start the main part of the manuscript here.
%%%%%%%%%%%%%%%%%%%%%%%%%%%%%%%%%%%%%%%%%%%%%%%%%%%%%%%%%%%%%%%%%%%%%
\section{Introduction} 

\textbf{Introduction}

Amyloid fibrils, generated by the aggregation of misfolded protein, cause many
neurodegenerative diseases such as Alzheimer's disease, Parkinson's disease,
type 2 diabetes, Huntington's disease, transmissible spongiform encephalopathies
etc\cite{Chiti2006}. Out of all these neurodegenerative diseases, Alzheimer's
disease becomes the most common in recent era\cite{..}.  Fibrillization of
A$\beta$40 peptide \cite{Hardy2002, Selkoe2004} is mainly responsible for
Alzheimer's disease. There are many different stages of A$\beta$40 amyloid
fibril which act as a toxic substance. Initially it was thought that mature
fibril is the main source of toxicity \cite{Hardy2002}.  Recent research shows
that small soluble oligomer is more toxic than matured fibril \cite{Haass2007}.
There are many reports regarding the inhibition of fibril formation, as example
amphiphilic surfactant \cite{Steven2005}, polyphenols\cite{Porat2006},
quinone-tryptophan hybrid (NQTrp)\cite{Attali2010}, nanoparticle coated with
histidine-based polymer\cite{Palmal2014} are capable to inhibit amyloid fibril
growth. But still there is a serious lack of understanding how to destabilize a
matured fibril or a small oligomer.

In the past decades, nanotechnology has been developed tremendously.  Nanoscale
materials have been used in the medicinal field such as drug delivery, in vivo
imaging, invitro diagnostics to prevent various diseases
\cite{Bianco2005,Lademann2007,Igor2005, Paul2004,Pearson2016}. Applications of
nanoscale systems in medicinal field make investigators interested to explore
the interactions of nano systems with biological molecules like lipid-bilayer
\cite{Hao2014,Lu2014}, protein molecules \cite{Shemetov2012,Nel2009}.  Recently
it is found that nanoparticles (NPs)  can be used as broad spectrum virucidal
drugs \cite{Francesco}. So it is very attractive to investigate the interactions
of different types of NPs with A$\beta$40 amyloid fibril from the perspective of
possible therapeutic approach.

Experiments have demonstrated that in some cases, NPs induce fibrillization of
the peptides \cite{Linse2007,Anika2016} and in other cases, they prevent
\cite{Celia2008,Yoo2011}. Linse et al.\cite{Linse2007} have shown that the
induced fibrillization depends on size, shape and the hydrophobicity of the NP
surface whereas Yoo et al. \cite{Yoo2011} have suggested that the inhibition of
fibrillization is the result of strong van der Waals interactions between the
NPs and the fibril.  Molecular dynamics study by Brancolini et al.
\cite{Brancolini2015} has shown the influence of citrate ligated gold NPs on an
amyloidgenic protein. Recently Yang et al. \cite{Yang2015} have modeled the
disruption of amyloid fibrils by graphene.   

In this work, we model five different NPs containing $2.2$ nm diameter core and
$90$ ligands. Each NP was simulated in the presence of the fibril to study the
NPs and fibril interactions.  To find the nature of the interactions, detail
calculations of coupling (Coulombic coupling and van der Waals interaction)
between the NPs and fibril are performed. Then we systematically explore the
effects of NPs interactions on the fibril by calculating the relative root mean
square deflection of the backbone of $\beta$ sheet with respect to the simulated
fibril without any NP, measuring twist angles between two peptides of the
fibril, compairing the number of salt bridge interactions and heavy atoms
contact in the fibril in the presence or absence of NPs. At the end, we also
calculate the overall potential energy of the fibril $\beta$ sheet with or
without the presence of NP to compare the change of stability in the fibril
$\beta$ sheet.  Finally we investigate the capability of the NPs to inhibit the
growth of the amyloid fibril by simulating the NPs at the tip of the fibril.

\section{Results and Discussion}

\textbf{Results and Discussion}

The NPs are modeled with five different ligands: positively charged
($NH_{3}^{+}$ terminal group) (pos-lig), negatively charged (neg-lig)
($SO_{3}^{-}$ terminal group, neutral (nut-lig) (NQTrp terminal group) and
peptide ligand (pep-lig) (Fig.~\ref{ligs}).  Except pep-lig, the other ligands
contain PEG (polyethylene glycol) chain to remove undesired protein
interactions and to increase the solubilty of the NPs. The PEG chains are short
(only two polyethylene glycol units) to achieve coherent interactions and to
remove the interligand interactions \cite{Hao2014}.  In the case of pep-lig, we
design the ligand (Glu-Leu-Val-Phe-Phe-Ala-Lys-Lys) following the aminoacid
sequence (Lys-Leu-Val-Phe-Phe-Ala-Glu-Asp) of $\beta$ sheet region of the
fibril where we replace the charged amino acids with oppositely charged amino
acids keeping the remaining same. The peptide ligands are attached with the
metallic core by thiol group of Cysteine amino acid. 
 
\begin{figure}[h] \centering
	\includegraphics[width=8cm,height=5cm,keepaspectratio]
	{ligs} \caption{(Pos-lig) PEG chain ternimated with $NH_{3}^{+}$
group, (Neg-lig) PEG chain terminated with $SO_{3}^{-}$ group, (Nut-lig) PEG
chain terminated with NQTrp (quinone-tryptophan hybrid), (Pep-lig)
Peptide ligand}\label{ligs} \end{figure}

\begin{table}[]
\centering
\caption{Detail ligands ratio of each nanoparticle}
\label{NPstable}
\begin{tabular}{lllll}
	Pos    & 	90 pos-lig	\\
        PosNQ  & 	80 pos-lig, 10 nut-lig	\\
        NegNQ  & 	80 neg-lig, 10 nut-lig	\\
        PepNP  &        90 pep-lig	\\
        Janus  &        45 pos-lig, 45 neg-lig	
\end{tabular}
\end{table}

Total five different types of NPs (Table.~\ref{NPstable}) are prepared to
explore the interactions between NPs and A$\beta$40 amyloid fibril.  All the NPs
contain $2.2$ nm diameter gold core. $90$ ligands are uniformly distributed on
gold surface. Since A$\beta$40 amyloid fibril is overall negatively charged, we
model three different types of NPs with overall postive charge. In the case of
PosNQ and NegNQ, the number of nut-lig is optimized in such a way that multiple
NQTrp molecules cannot make cluster on the NP surface. The clustering of these
NQTrp groups prevent them to interact with the amino acids of the
fibril\cite{Pearson2016}.  Increasing the number of nut-ligs also decreases the
solubility of the NPs in water. In the case of janus NP, there are $45$ positive
ligands at one side of the NP and in the other side, there are $45$ negative
ligands.

%\begin{table}[]
%\centering
%\caption{Detail ligands ratio of each nanoparticle}
%\label{NPstable}
%\begin{tabular}{lllll}
%	Pos    & 	90 pos-lig	\\
%        PosNQ  & 	80 pos-lig, 10 nut-lig	\\
%        NegNQ  & 	80 neg-lig, 10 nut-lig	\\
%        PepNP  &        90 pep-lig	\\
%        Janus  &        45 pos-lig, 45 neg-lig	
%\end{tabular}
%\end{table}

\begin{figure}[h] \centering
	        \includegraphics[width=6cm,height=10cm,keepaspectratio]
		{posNQ} \caption{(Top) Initial structure of PosNQ NP with fibril, (Bottom)
shows the NP interacting with L1 layer of the fibril after $90$ ns of
simulation}\label{PosNQ} \end{figure}


In our simulations, the positively charged NPs (Pos, PosNQ, PepNP) are strongly
interacting at $\beta$ sheet surface  of layer 1 (L1) of the fibril. In the
$\beta$ sheet region, Pos NP mainly becomes close to the negatively charged
Glu22 and Val24. With time, Asp1, Ala2, Glu3 and Arg5 of the random chain region
of L2 layer also starts to interact with the positive surface of the fibril.
These interactions generate a strong Coulombic coupling (average Coulombic
interaction energy $-290$ kcal/mol) between Pos and the fibril whereas the van
der Waals (vdW) interaction strength is close to zero (average vdW interaction
energy $13.0$ kcal/mol) Fig.~\ref{interaction}. 

\begin{figure}[h] \centering
	        \includegraphics[width=6cm,height=10cm,keepaspectratio]
		{interaction} \caption{Electrostatic, van der Waals
and total nobonding interaction energy}\label{interaction} \end{figure}

In the case of PosNQ NP, there are two different types of terminal groups,
positively charged ammonium groups and neutral NQTrp groups. The positively
charged groups are mainly interacting with Phe20, Glu22 and Val24.
Fig.~\ref{interaction} shows that Coulombic interactions between positively
charged groups and amino acids provides a strong stabilization energy (average
Coulombic interaction energy $-266.0$ kcal/mol).  Though the Coulombic
interaction energy is comparatively smaller than Pos, there is a very strong
vdW interaction energy (average strength $-90.0$ kcal/mol) due to the addition
of NQTrp molecules at the terminal of $10$ ligands. Out of $10$ NQTrp groups,
$4$ are close to the $\beta$ sheet region and are interacting with Hse14, Gln15,
Lys16, Leu17, Val18, Phe20, Val24, Gly25, Ser26 and Asn27. Few random chains of
the fibril are also wrapping the NP. The amino acids of the random chains close
to the NP are Asp1, Glu3, Arg5, Asp7, Ser8, Glu9, Tyr10, Gln11, Hse13.

The length of the ligand of PepNP is longer than the other ligand. It increases
the overall radius of the NP bigger than all other NPs.  The Coulombic coupling
(average strength $-185.0$ kcal/mol) and the vdW interaction strength (average
vdW interaction energy $-135.0$ kcal/mol) between the NP and fibril are almost
comparable to each other.  PepNP is strongly wrapped by the random chains of the
fibril.  It can be seen that in the total interaction energy major contribution
comes from the random chain region of the fibril (average Coulombic coupling
between the NP and random chains $-122.0$ kcal/mol and average vdW interaction
strength $-105.0$ kcal/mol) which is more than $75$\% of total interaction
energy. The interacting amino acids are Asp1, Glu3, Phe4, Arg5, Hse6, Asp7,
Ser8, Gly9, Tyr10, Glu11, Val18, Phe19, Phe20, Glu22, Asp23 and Val24.  

In the case of Janus NP, a side of the NP is completely positive whereas the
other side is completely negative. The positive side of the NP mainly interacts
with the Phe20, Glu22, Asp23 and Val24 amino acids of $\beta$ sheet region
whereas the negative side interacts with Hse13, Lys16, Val18 and Phe20. The
random chains of the fibril does not wrap the NP. Our calculation shows that
there is an attractive Coulombic coupling between the NP surface and $\beta$
sheet region of the fibril. But the random chians increase the Coulombic
repulsion. As a whole, the Coulombic interaction strength is almost zero and vdW
interaction strength is $-46.0$ kcal/mol. 

We have also modeled a negatively charged NP, named NegNQ which shows a strong
vdW interaction (-$60.0$ kcal/mol) due to the presence of the NQTrp molecules.
But there is stronger repulsive Coulombic interactions ($200.0$ kcal/mol)
between the NP and the surface of the fibril which prevents the NP to be
properly nested on the fibril surface. 

\begin{figure}[h] \centering
	        \includegraphics[width=8cm,height=8cm,keepaspectratio]
		{new-rmsdB} \caption{Root mean square deviation of the peptide backbone 
of L1 layer of the fibril of each NP-fibril system and only-fibril
system with respect to the fibril without NP}\label{rmsdB1} \end{figure}

We explore the effects of binding of four NPs (except NegNQ) on the fibril
surface since NegNQ does not bind on the surface.  At the beginning, we compare
the root-mean-square-deviation (RMSD) of the backbone atoms of the fibril with
respect to the simulated fibril without NP. Fig.~\ref{rmsdB1} shows the RMSD of
the L1 layer (layer close to the NP). In the same plot, the RMSD of the backbone
atoms of $\beta$ sheet region of only-fibril system is also shown for last $5$
ns (total $100$ ns simulation) with respect to the same reference frame used for
other NP-fibril systems. The plot shows that the RMSDs of NP-fibril systems are
always bigger than only fibril system. It proves that binding of NP induces some
rearrangement in the $\beta$ sheet region.  RMSD is maximum for the case of
PosNQ which can be explained by the strongest binding of PosNQ on $\beta$
surface of the fibril. For Pos and PepNP, RMSDs of fibril $\beta$ sheet are also
correlated with the binding strength. Though binding strength of Janus NP is
significantly smaller than Pos, PosNQ and PepNP, the RMSD of $\beta$ sheet is
similar with other NPs. The unique charge distribution on the surface of Janus
NP might be the reason. It induces some rearrangement in the backbone close to
the NP since the positive and negative both types of amino acids are interacting
with Janus NP.

The RMSD of the backbone of loop region (Ser26 to Ile31) is also performed
separately. In all the cases we observed that RMSD of the loop (the average RMSD
of loop for Pos, PosNQ, PepNP and Janus are $4.07$, $5.16$, $4.445$ and $4.63$
{\AA}  respectively)is smaller than the RMSD of remaining $\beta$ sheet region
(the average RMSD of the remaining $\beta$ sheet residues for Pos, PosNQ, PepNP
and Janus are $4.65$, $6.04$, $5.28$ and $5.05$ {\AA} respectively). This result
shows that the influence of the NPs is lower on the dynamics of the loop region
than the remaining part of the $\beta$ sheet. 

\begin{figure}[h] \centering
	        \includegraphics[width=8cm,height=8cm,keepaspectratio]
		{new-twist-com-a} \caption{(A) Average twist angles of the
		peptides for each frame of Pos, PosNQ, PepNP and
		Janus. (B) The distrbution of the twist angles of the 
peptides for each type of NP}\label{twist-com} \end{figure}

RMSD result shows that the backbone of fibril $\beta$ sheet is more deviated by
the presence of NPs bound on the surface. So it is highly expected that NP
binding might induce some structural change in the fibril.  To investigate the
change in structural property of the fibril, we measure the average twist angle
of the peptides in the presence and the absence of NP. In the case of NP-fibril
systems, the average twist angle of the L1 layer of the fibril is more than
$7.6$ degrees (Pos: $7.7$, PosNQ: $7.93$, PepNP: $8.08$ and Janus: $7.62$)
whereas the average twist angle of the same layer for only-fibril system is
$6.48$ degree (Fig.~\ref{twist-com}A).  Fig.~\ref{twist-com}B shows the
histogram of the twist angles for each system. In the histogram, total number of
counts for each system is $10500$ (total $500$ frames and each frame contains
$21$ angles).  In all the cases, most of the twisting angles belong to $0$ to
$10$ degree (Only fibril: $85.36$\%, Pos: $82.5$\%, PosNQ: $83.8$\%, PepNP:
$74.5$\%, Janus: $75.5$\%). But the size of the distribution tail is longer in
the presence of the NPs. The highest twist angle is less than $24$ degree for
only fibril system which is also very similar for Janus NP.  But in the cases of
Pos, PosNQ and PepNP, the highest twist angle is close to $30$ or more than $30$
degree. The percentages of twist angle equal or more than $20$ degree for only
fibril, Pos, PosNQ, PepNP and Janus are $1.4$\%, $9.3$\%, $8.7$\%, $8.6$\% and
$1.13$\% respectively. In case of Janus, the percentage of twist angle in
between $10$ to $20$ degree is the highest ($23.33$\%) among all the systems.  

To investigate the reason behind the increment of average twist angles in the
presence of NP binding, we spot out the positions of the large twist angles. We
find that the peptides close to the NPs are arranging themselves in such a way
that they can interact with the NPs and at the same time can make inter peptide
interactions.  There are at least two positions in the fibril, where the set of
peptides close to the NP meet other peptides. The large twist angles are present
at the junction point and it increases the average twist angles of L1 layer.

%\begin{figure}[h] \centering
%	        \includegraphics[width=8cm,height=8cm,keepaspectratio]
%		{new-twist-com-a} \caption{(A) Average twist angles of the
%		peptides for each frame of Pos, PosNQ, PepNP and
%		Janus. (B) The distrbution of the twist angles of the 
%Peptides for each type of NP}\label{twist-com} \end{figure}

We find that there is a direct impact of NP binding on the twist angles of the
fibril. Another very important characteristic of fibril self-assembly is the
formations of salt-bridge and hydrophobic interactions which provide a high
stabilization in enthalpy and compensate the loss of entropy in the formation
process.  To examine the effect of NP binding on the stability of fibril, we
measure the number of salt bridge interactions and the number hydrophobic
contacts in $\beta$ sheet region of the fibril. We calculate the number of salt
bridge interactions with measuring the number of carboxylate oxygen atoms of
Aspartic acid and Glutamic acid within $4$ of {\AA} of ammonium nitrogen of
Lysine and guanidine nitrogen of Arginine. In the case of only fibril system,
the average number of salt bridge interaction is $94.23$. In the presence of
NPs, the average interactions vary in between $90-105$ (fig.~\ref{contact}A). We
found that PosNQ has the lowest number of interactions ($90.99$) whereas Pos has
the highest ($104.09$).  The number of salt bridge interactions for Janus and
PepNP are $91.67$ and $96.26$ respectively. Combining all these information, it
can be infered that binding of NP does not change significantly the number of
salt bridge interactions.

\begin{figure}[h] \centering
	        \includegraphics[width=8cm,height=8cm,keepaspectratio]
		{contact} \caption{(A) Number of salt bridge interactions in 4th to 25th 
  peptide of the $\beta$ sheet in both layers. (B) Total number of heavy 
  atom contacts in the $\beta$ sheet of the fibril (4th to 25th peptide).
  (C) (Left) Number of heavy atom contacts in each layer L1 and L2 separately.
  (Right) Number of heavy atom contacts between L1 and L2 layer.}\label{contact} 
\end{figure}

To calculate the total number of hydrophobic contacts in the $\beta$ sheet of
the fibril, we measure the number of interpeptide contacts among all the heavy
atoms with a cut-off distance $4$ {\AA} (Fig.~\ref{contact}B). We separately
identify all the interpeptide heavy atoms contacts for both layers of the fibril
(Fig.~\ref{contact}C(Left)). Then we measure the number of contacts between two
layers (Fig.~\ref{contact}C(Right)). Fig.~\ref{contact}B shows that except
PepNP, for all other NPs total number of heavy atoms contact decreases with
respect to only fibril system.  This reduction is mainly observed in L1 layer
(Fig.~\ref{contact}C. The strong interactions of the NPs with L1 layer of the
fibril are mainly responsible to decrease the number of contact points in L1
layer.  In the case of PosNQ, the contact in L1 layer is minimum. It shows that
the mixture of positive and nutral ligands terminating with NQTrp reduces the
maximum number of interpeptide contact points which definitely reduces the vdW
potential energy among the fibril $\beta$ sheet. Pos NP decreases the intralayer
contact points for both layers. But the number of interlayer contact points is
almost same like only fibril system. The behavior of Janus is little different
from Pos. The numbers of intralayer contact points for L1 and L2 are little
bigger than Pos, but the number interlayer contact points is significantly lower
than only-fibril system and Pos. Eventually, in the case of Janus NP, fibril
contains the second lowest number of heavy atom contacts after PosNQ NP. The
presence of unique charge distribution on the surface of Janus NP attracts both
positive and negative amino acids of the fibril which may be the reason of
significant reduction of interlayer heavy atom contacts by Janus NP.  In the
case of PepNP, the total number of heavy atom contacts increases which indicates
that the fibril becomes stabilize in the presence of more peptide concentration.
This result is quite consistent with Sato et al's \cite{Sato2007}study where
they engineer peptides according to sequence $A\beta14-23$. The peptides were
used to capture soluble peptide oligomer and form amyloid like fibril.

%\begin{figure}[h] \centering
%	        \includegraphics[width=8cm,height=8cm,keepaspectratio]
%		{contact} \caption{(A) Number of salt bridge interactions in 4th to 25th 
%  peptide of the $\beta$ sheet in both layers. (B) Total number of heavy 
%  atom contacts in the $\beta$ sheet of the fibril (4th to 25th peptide).
%  (C) (Left) Number of heavy atom contacts in each layer L1 and L2 separately.
%  (Right) Number of heavy atom contacts between L1 and L2 layer.}\label{contact} 
%\end{figure}

%of the NP and normalize the number with total number of atoms of the fibril
%within this region. We perform the same calculation for the atoms outside $1$ nm
%of the NP. The result shows that in a normal A$\beta$ amyloid fibril there are
%$5.03$ H-bonding per $1000$ atoms.  But in the local region of the NP, this
%number decreases to $1.7$ to $2.8$ in the case of Pos, PosNQ and Janus NP. In
%the local region of Pos, PosNQ and Janus NP, the number is $1.7$, $2.8$ and
%$2.2$ respectively whereas outside this region, the number of H-bonding per
%$1000$ atoms are very close (Pos: $4.6$, PosNQ: $5.1$ and Janus: $4.8$) to the
%normal fibril. We do not find any difference in H-bonding density (Local region
%of PepNP: $5.1$ and Outside this region: $4.4$) in the local region of PepNP. 
%
%These results show that binding of Pos, PosNQ and Janus NP reduce the H-bonding
%density almost to the half in their local regions. PepNP cannot affect the
%H-bonding network in its neighborhood. Damage of this H-bonding network might be
%related with the curvature of the NPs.  For Pos, PosNQ and Janus, the curvature
%of the NP is bigger than the curvature of PepNP bacuse of having smaller ligands
%compared to the long peptide ligands. To make strong interaction with smaller
%NPs (Nps with higher curvature), the fibril needs to follow the surface
%structure of the NPs which can reduce the overall number of H-bonding in the
%fibril close to the NP.

To shed light on the change of overall stability of the fibril due to the
interactions with the NPs, we calculate Coulombic and vdW potential energy of
the $\beta$ sheet of the fibril from 4th to 25th peptide for both layers.  In
the calculation, we remove few peptides in both sides because the fluctuation of
the peptides at the side is more than the inner peptids which might affect the
result of overall stabilization.  Fig.~\ref{internal-beta} shows the change of
potential energy of the $\beta$ sheet region in the presence of each NP compared
to the fibril without any NP. 

\begin{figure}[h] \centering
	        \includegraphics[width=7cm,height=7cm,keepaspectratio]
		{internal-beta} \caption{Change of Electrostatic, 
van der Waals and total nobonding energy of the internal energy of 
$\beta$ sheet region of the fibril}\label{internal-beta} \end{figure}

The calculation shows that the $\beta$sheet region of A$\beta$40 amyloid fibril
becomes stabilized mainly by vdW interaction strength (~$-3900$ kcal/mol).
Compared to vdW, there is a weak repulsive Coulombic interaction (~$60$
kcal/mol) which is due to the presence of same charged amino acids in a closely
linear arrangement. Fig.~\ref{internal-beta} shows the effect of all four NPs on
the internal energy of the fibril $\beta$ sheet.  There is a very small change
($-4.7$ to $7.8$ kcalmol) in Coulombic coupling which is expected from the
calculation of the number of salt bridge interaction.  In all the cases (In the
presence and the absence of NP), the average number of salt bridge interaction
is in between $90-105$. According to the potential energy of fibril $\beta$
sheet, Coulombic coupling becomes destabilized for all the NPs except Janus NP.
But this change is significantly smaller than vdW interaction strength. vdW
interaction strength is destabilized by all the NPs.  The lowest vdW interaction
change is observed for PepNP ($45$ kcal/mol) and the highest is for PosNQ
($285.8$ kcal/mol). The trend makes a good correlation with the measured number
of heavy atoms contact calculated using $4$ {\AA} cutoff except PepNP. In the
case of PepNP, though number of contacts among heavy atoms is increasing in $4$
{\AA} cut-off, still vdW interaction strength becomes destabilized. It means
though the number of heavy atoms contact increases in the close range ($4$
{\AA}), it decreases in the long range ($4-10$ {\AA}).  For Pos, the change in
vdW interaction strength ($96.4$ kcal/mol) is not so big like PosNQ and Janus.
Binding of Pos with fibril surface occurs mainly due to the Coulombic coupling.
It changes the structure of the fibril close to the NP. But the peptides of the
fibril rearrange themselves in such a way that Pos cannot destabilize the fibril
$\beta$ sheet in a large extent. In the case of PosNQ and Janus, we can observe
a large change of vdW interactions towards the positive direction. The $\beta$
sheet region of the fibril becomes destabilized by $285.8$ kcal/mol for PosNQ
and $214.2$ kcalmol for Janus. In the case of PosNQ, the NQTrp molecules bring
extra interactions with many amino acids of the $\beta$ sheet (Hse14, Gln15,
Lys16, Leu17, Val18, Phe19, Phe20, Glu22, Asp23, Val24, Gly25, Ser26, Asn27)
which are very similar like Zhang et. al and Scherzer-Attali's study
\cite{Attali2010, Zhang2014}.  These extra interactions may destabilize the
$\beta$ sheet. For Janus, due to the presence of unique surface charge
distribution (one side positive and the other side negative) the amino acids of
$\beta$ sheet are trying to interact with both sides of the NPs.  In one side,
Lys16 amino acid can interact with negative ligands whereas in other side Glu22,
Asp23 can interact with positive ligands. These combined interactions
destabilize the fibril $\beta$ sheet. 
  
\begin{figure}[h] \centering
	        \includegraphics[width=7cm,height=7cm,keepaspectratio]
		{interaction-tip} \caption{Electrostatic, van der Waals
and total nobonding interaction energy of each nanoparticle
interacting at the tip of the fibril}\label{interaction-tip} \end{figure}
 
In addition to all, we investigate the interactions of all five NPs at the tip
of the fibril to figure out if any of these NPs be potentially able to stop the
growth of the fibril. We calculate the interaction energy between the NPs and
the fibril in the same way like before.  Fig.~\ref{interaction-tip} shows that
the positively charged NPs experience strong repusive electrostatic interactions
at the tip of the fibril. The presence of Nut-ligs containing NQTrp groups and
the pep-ligs are able to produce stabilized vdW interactions. But vdW
interactions are not enough to compensate the repulsive electrostatic
interactions. In the case of Janus NP, we simulated three different systems
according to the orientation of the NP (Positively charged side towards fibril,
negatively charged side towards fibril, positive and negative both oriented
towrds the fibril) towards the fibril due to an assymetric ligand distribution.
In the fig.~\ref{interaction-tip}, the average of all three simulations are
shown for Janus NP. Only NegNQ shows strong attractive interaction energy,
average electrostatic interaction energy $-108.6$ kcal/mol and average vdW
interaction energy $-24.1$ kcal/mol.  It can be explained by the equipotential
surfaces of $A\beta40$ fibril shown in figure SI.  The light blue color surface
at the tip of the fibril corresponds to $2.6$ V whereas the light pink surfaces
on $\beta$ sheet region correspond to $-10.4$ V.  It shows that negatively
charged NPs are more preferable to block the growth of the fibril.

%\begin{figure}[h] \centering
%	        \includegraphics[width=7cm,height=7cm,keepaspectratio]
%		{interaction-tip} \caption{Electrostatic, van der Waals
%and total nobonding interaction energy of each nanoparticle
%interacting at the tip of the fibril}\label{interaction-tip} \end{figure}


\section{Conclusion} 

\textbf{Conclusion} In summary we have studied five different NPs interacting
with $A\beta40$ amyloid fibril. We found that positively charged NPs can
interact with fibril surface whereas negatively charged NP can interact at the
tip of the fibril.  In the case of positively charged NP, addition of NQTrp
inhibitor at the terminal of few PEG ligands increases the vdW contact with
fibril amino acids which decreses the interpeptide vdW contacts in the fibril
and eventually destabilize the fibril $\beta$ sheet. Though Janus NP also can
destabilize the fibril $\beta$ sheet, PosNQ shows maximum ability in
destabilization. Our detail simulations provide some atomistic understanding how
the NPs are destabilizing small fibril or peptide oligomers. This study is
extremely helpful for experimentalists to engineer an effective system to break
the peptide oligomer. Still there are many questions which need to be resolved.
First of all what is the optimum concentration of NQTrp terminated ligand to get
maximum destabilization in the fibril? There are still many fibril inhibitors
like polyphenol, amphiphilic surfactant etc. It would be quite interesting to
figure out which one affects maximum in this environment.  What will be the
effect of multiple PosNQ NPs on the same fibril?  How NegNQ NP affect the
stability of the fibril while interacting at the tip of the fibril?  In future,
we would like to continue our research in this field and try to resolve these
questions computationally. 



\section{Method}

\textbf{Methods}

We model the NPs using our own TCL code. The initial structure of A$\beta$40
fibril is based on pdb ID 2LMO \cite{Petkova2006}. In the coordinate file, first
eight amino acids of each peptide were missing. We use Modeller program
\cite{Sali1993,Fiser2000} to add the missing residues.  Then the fibril is
prepared with $29$ A$\beta$40 monomers. After preparing the fibril, we simulated
it in $150$ mM NaCl solution for $5$ ns. Then we placed the NP $5$ {\AA} above
the fibril. All the simulated systems contain $400,000-600,000$ atoms. After the
initial minimization and warming to $300$ K, ions and water molecules were
equilibrated for $2$ ns. During ion and water equilibration, the movement of NP
core and the protein backbone is restrained with a harmonic force constant of
$1$ kcal/mol{\AA}$^{2}$. After releasing the NP and protein, all the systems
were simulated for $90-100$ ns. Visual Molecular Dynamics (VMD)
\cite{Humphrey1996} is used to visualize the NPs and the fibril. 

We perform atomistic molecular dynamics (MD) simulations of each NP, fibril
system in $150$ mM NaCl solution. The systems are simulated with NAMD package
\cite{Philip2005} using CHARMM general\cite{Mackerell2010,Yu2012} and protein
forcefield \cite{Mackerell1998,Mackerell2004}. In the simulations, we use
Langevin dynamics with damping co-efficient of $\gamma_{lang} = 0.1$ $ps^{-1}$.
Nonbonded van der Waals (vdW) interactions are calculated with the Lennard-Jones
($12$,$6$) potential using the cut-off distance d = $10$ {\AA}.  Long range
electrostatic interactions are calculated by the PME method \cite{PME} and the
MD integration time step is set to $2$ fs. The simulations are performed using
NPT ensemble (P = $1$bar and T = $300$K) in the presence of periodic boundary
conditions. 

After simulating each system for $90-100$ ns, we measured the interactions
energy (electrostatic and vdW interaction energy) of the NPs with fibril 
using VMD energy analysis plugin.  During the electrostatic
calculations, we consider the dielectric constant of water equals to $78.5$. For
vdW interaction, we use ($12$,$6$) Lennard-Jones potential with $10$ {\AA}
cut-off. The overall potential energies of the $\beta$ sheet of the fibril in
the presence and the absence of NP are also investigated using VMD plugin. Then
we calculate the change of potential energy of the fibril $\beta$ sheet in the
presence of NP following the equation ${\Delta}E = E_{NP} - E_{Fibril}$ 
where $E_{NP}$ is the potential energy of fibril $\beta$ sheet in the presence
of NP and $E_{Fibril}$ is the potential energy of the same region of the fibril
of only fibril system. In this calculation, we use 4th to 25 th peptide out of
29 peptides of both layers of the fibril. All the energies are calculated for
last $200$ frames ($2$ ns). We also simulate five NPs at the tip of the fibril
for $4-5$ ns and calculate the interaction energy of the NP with fibril tip.

Root-mean-square-deviations (RMSD) of backbone atoms of the $\beta$ sheet region
for each NP-fibril system and only fibril system are calculated compared to the
system containing only the fibril. Before calculating the RMSD, the whole fibril
of NP-fibril system is alligned with the fibril of only fibril system (the
reference frame was taken after $30$ ns simulation of only fibril system). Then
we calculate the RMSD of each peptide separately and finally take the average
with the total number of peptides. The calculations are performed using equation
$RMSD = 1/n \sum_{j=4}^{25} d_j$, where $n$, $j$ and $d_j$ are the number
of peptides, the index of each peptide and  RMSD of j-th peptide respectively.
The RMSD is calculated for $\beta$ sheet of both layers of the fibril. In each
layer there are $29$ monomers. In the calculation, $4$th to $25$th monomers are
taken into account because of higher fluctuation of the terminal peptides. The
calculation is performed on last $500$ frames ($5$ ns)

%\begin{equation}
%	$RMSD = 1/n \sum_{j=4}^{25} d_j$
%\end{equation}
%Where n is the number of peptide. Here n is $22$. J is the index of
%each peptide and $d_j$ is RMSD of j-th peptide.

To calculate the average twist angle between two peptides, we first take a
vector from alpha carbon atom of $32$nd amino acid residue to $18$th residue of
each peptide. We measured the angle between the vector of one peptide and the
same vector of the next peptide. Then we calculate the average of the twist
angle. Like RMSD, we also calculate the twist angle from 4th to 25th monomer of
each layer (L1 and L2 layer separately). The calculation is performed on last
$5$ ns using equation $\theta_{twist} = 1/m \sum_{j=4}^{24} \phi_{j,j+1}$ 
where $m$ is the number of peptide pairs, $j$ and $(j+1)$ are the indexes of two
neighboring peptides and $\phi_{j,j+1}$ is the twist angle for a neighboring
pair of peptide.

Total number of salt bridge interactions is calculated by measuring the number
of carboxylate atoms of Aspartic acid and Glutamic acid within $4$ {\AA} of
ammonium nitrogen of Lysine and guanidine nitrogen of Arginine. We take the
average of the number of salt bridge interactions on last $100$ frames ($1$ ns).
We also calculate the total number of interpeptide heavy atoms contact of the
$\beta$ sheet region. First we investigate the number of contacts of each layer
individually, then we measure the number interlayer contacts. In both cases, we
use $4$th to $25$th monomers out of $29$ monomers of each layer. We use last
$50$ frames ($0.5$ ns) for this measurement. Number of intralayer heavy atoms
contact is calculated using $N_{L} = \sum_{i=4}^{25}\sum_{j=i+1}^{j=i+5}
N_{i,j}$ where $N_{L}$ is total number of contacts in L1 layer or L2 layer
and $N_{i,j}$ is number of contacts in i-th and j-th peptide of the same layer.
To measure the number of interlayer heavy atoms contact, we use $N_{L1,L2} =
\sum_{i=4}^{25}\sum_{j=i-2}^{i+2}N_{i,j}$, where $N_{L1,L2}$ is number of
contacts between L1 and L2 layer, $N_{i,j}$ is the number of contact between
i-th peptide of L1 layer and j-th peptide of L2 layer. 



   
%The number of H-bonding in the peptides close to the NP is compared with the
%peptides far from the NP. We measure the number of atoms and number of H-bonding
%in the fibril within $1$ nm and outside $1$ nm of the NP. The conditions imposed
%to determine a H-bonding are donor-acceptor distance (less than $3$ {\AA}) and
%the angle between donor-H vector and H-acceptor vector (less than $20$ degree.
%We show the number of H-bonding per $1000$ atoms. The H-bonding was analyzed for
%last $5$ ns ($500$ frames) 
%


























%We use atomistic molecular dynamics (MD) simulations to model the
%interactions of the NPs with A$\beta$40 fibril. The systems are
%simulated with NAMD package \cite{Philips2005}.  We model multiple NPs
%with four different types of the ligands (Figure1) at the surface.
%Two different kinds of PEG ligands are prepared. In most of the PEG
%ligands, positively charged ammonium groups are at the end and in few
%cases, 1,4-napthoquinon-2-yltryptophan (NQTrp) group is used as the
%terminal group. NQTrp is a potential inhibitor \cite{..} for the
%self-assembly of A$\beta$40 fibril. We also design a peptide ligand
%with sequence CELVFFAKK which is complementary to residue $16$ to $23$
%(KLVFFAED) of A$\beta$40 peptide. The diameter of the gold core of the
%NP is $2.2$nm. We prepare three different NPs, 1) $90$ positively
%charged PEG ligands (pos), 2) $80$ positively charged PEG ligands and
%$10$ PEG ligands terminated with NQTrp (pos-NQTRP) and 3) $90$ peptide
%ligands (np-peptide). The ligands are evenly distributed on the
%surface of the core.  The initial structure of A$\beta$40 fibril is
%based on the pdb id 2LMO \cite{Petkova2006}. We add the missing
%residues in the fibril using Modeller program
%\cite{Sali1993,Fiser2000}. Then the fibril is propagated upto $15$ nm.
%
%Initially we simulate the particles and the fibril individually in an
%ionic solution ($0.15$ M NaCl solution) for $10$ns. After that, each
%type of the NPs is placed on the fibril and the combined system is
%kept in an ionic solution with physiological concentration ($0.15$M).
%Each combined system is simulated for 60 ns. The simulation of only
%fibril is also ontinued upto 40 ns. In all the simulations, we use
%CHARMM general \cite{Mackerell2010,Yu2012} and protein forcefield
%\cite{Mackerell1998,Mackerell2004} for the simulations. The
%simulations are performed in an NPT ensemble at T=$300$ K and P=$1$
%bar. Nonbonding interactions are calculated using a cut-off distance
%of $d=10$ angstrom and long range electrostatic interactions are
%calculated by the PME method \cite{Darden1993} in the presence of the
%periodic boundary conditions. The systems are simulated using the
%Langevin dynamics with a damping constant of $0.1 ps^{-1}$ and the
%timestep of $2$ fs. During the simulations, we applied a harmonic
%potential to prevent the translational and rotational motion of the
%fibril. 
%
%We calculate Coulombic
%coupling and van der Waals interaction strength of the NPs with fibril
%over the whole trajectory ($100$ns). In the calculations of Coulombic
%interaction strength, the dielectric constant is equal to $78.5$.
%
%Initially we model three different kinds of NPs, neutral, negative and
%positive. In the MD simulations, at the beginning, the NPs are kept
%within $1$nm distance from the $\beta$ sheet of the fibril. With time,
%the NPs become close to the fibril. Fig.~\ref{particles}\,A shows that
%the neutral NPs become stabilized either on the the freely moving
%amino acid chains or on the $\beta$ sheet. The negatively charged NPs
%are initially nested on the positively charged K$16$ residues of the
%fibril. After nesting, they are moving towrads the edge and become
%stabilize on the freely moving aminoacid chains
%(Fig.~\ref{particles}\,B). In case of the positively charged NPs, the
%particles become stabilized over negatively charged E$22$ residues of
%the fibril (Fig.~\ref{particles}\,C). 
%
%Fig.~\ref{particles}\,D shows the van der Waals (vdW) and Coulombic
%coupling strength between the NPs and the fibril. For the neutral
%particles, NPs become stabilized by strong vdW interaction strength
%irrespective of their positions. The vdW interaction strength is close
%to $250$ kcal/mol. The Coulombic coupling strength is zero. In the
%case of the negatively charged particles, the total interaction
%strength varies $60$-$90$ kcal/mol where the major contribution comes
%from vdW interactions (vdW interaction strength:$40$-$80$ kcal/mol,
%Coulombic interaction strength:$10$-$30$ kcal/mol). The interactions
%strength for the positively charged particles belongs to $30$-$90$
%kcal/mol. But in this case, the Coulombic coupling strength
%($30$-$100$ kcal/mol) is playing the main role. The vdW interaction
%provides a small repulsive interaction ($5$-$10$ kcal/mol). 
%
%\begin{figure}
%\centering
%\includegraphics[width=3.5in]{parts1d-int5a.ps}
%\caption{\normalsize A) Neutral NPs,
%B) Negative NPs, C) Positive NPs. Fibril is shown using surface
%representation. Red represents negative, blue represents positive,
%green represents polar and white represents non polar aminoacids.
%In A, orange and yellow enclosure show K$16$ and E$22$ residues 
%respectively. The region between two black arrows is $\beta$ sheet
%region and the region between black and red arrow in both sides
%contains freely moving amino acid chains. D) Coulombic interactions, 
%van der Waals interactions and total interaction strength in kcal/mol}
%\label{particles}
%\end{figure}
%
%The neutral NPs can be properly wrapped up by the freely moving amino
%acid chains which increases the vdW interaction strength. The same is
%true for negatively charged NPs.  The K$16$ (Fig.~\ref{particles}\,A,
%the black circle) residues on which negatively charged NPs are nested,
%are very close to the freely moving chains. But the positively charged
%NPs are nested on E$22$ (Fig.~\ref{particles}\,A, the whitish
%circle)residues which are far from the free chains. That is the reason
%why the free chains can not properly wrap up the positively charged
%NPs. In the case of the negative particles, the Coulombic coupling is
%initially being stabilized, then it passes through a minimum point and
%after that it slowly becomes destabilized (Figure in supporting info).
%The destabilization appears due to the Coulombic repulsion of negative
%ligands with negative E$22$ residues and the negative residues of free
%aminoacid chains (D$1$, E$3$, D$7$, E$11$ residues).     
%
%\begin{figure}
%\centering
%\includegraphics[width=2.6in]{mixed2-int6a.ps}
%\caption{\normalsize A) NPs with $30$\% negative and $70$\%
%neutral ligands, B) NPs with $70$\% negative and $30$\% neutral
%ligands, C) Interaction strength of all four NPs with fibril
%in kcal/mol. Neg$30$\%-$1$ is particle $1$ of A, Neg$30$\%-$2$ is
%particle $2$ of A, Neg$70$\%-$1$ is particle $1$ of B and
%Neg$70$\%-$2$ is particle $2$ of B}
%\label{mixed}
%\end{figure}
%
%In the case of the negative particles, to increase the vdW
%interactions strength we substitute some of the negative ligands with
%neutral ligands. So we prepare two more different types of the
%particles ($30$\% negative ligands, $70$\% neutral ligands and $70$\%
%negative ligands, $30$\% neutral ligands). Fig.~\ref{mixed}\,A,B show
%that there are two different positions where these NPs can be
%stabilized. In both cases, initially the NPs are nested on the K$16$
%residues. Then either they are moving towards the freely moving amino
%acid chains or towards the $\beta$ sheet region
%(Fig.~\ref{mixed}\,A,B). From the energetics (Fig.~\ref{mixed}\,C), it
%can be observed that substituting some negative ligands with neutral
%ligands is increasing the vdW coupling strength ($150-250$ kcal/mol).
%But it adds a small repulsive Coulombic interactions ($25-50$
%kcal/mol) due to the Coulombic coupling between negative ligands and
%the negative amino acids (E$22$, D$1$, E$3$, D$7$, E$11$ residues) of
%$\beta$ sheet and free amino acid chains. The Coulombic coupling
%passes through a minimum point like fully negative particles.  With
%increasing the percentage of the negative ligands, the NPs on the
%$\beta$ sheet region become destabilized due to the presence of the
%strong Coulombic repulsion between the negatively charged ligands and
%the E$22$ residues 
%
%To understand the effect of the particle size on the interactions of
%the NPs with the fibril, we prepare a NP with a core of $5.5$ nm
%diamter and $400$ ligands ($70$\% negative ligands and $30$\% neutral
%ligands). The particle is nested on the fibril in the simulations.But
%the energetics (Fig.~\ref{large}\,A) show that it is not stable over
%the fibril. Though the attractive vdW interaction strength is $50$
%kcal/mol, the repulsive Coulombic coupling is $75$ kcal/mol (figure in
%supporting info) which destabilizes the particle on the fibril.
%
%\begin{figure}
%\centering
%\includegraphics[width=2.6in]{large2.ps}
%\caption{\normalsize A) Large nanoparticle with $70$\% negative and
%$30$\% neutral ligands on A$\beta$40 fibril, B) Pinkish surface
%shows the negative equipotential surface of potential $-8.2$V}
%\label{large}
%\end{figure}
%
%We perform an electrostatic surface calculation. Fig.~\ref{large}\,B
%shows that there are two parallel negative equipotential ($-8.2$ V)
%surfaces present in A$\beta$40 fibril. Only a small NP ($2-3$nm
%diameter) can be fit in between these two surfaces. There is a huge
%overlap between the negatively charged, large NP ($5-6$nm diameter)
%and the negative surface which destabilizes the nesting of large NP on
%A$\beta$40 fibril.
%
%In summary, we have provided a detailed study of the interactions of the
%NPs with A$\beta$40 fibril. Our study reveals that small NPs ($3$ nm
%diameter) can make stable interactions with the fibril regardless of
%their electrostatic property of the surface. The nesting positions of
%the NPs and the mode of interactions between NPs and the fibril are
%different depending on the surface property. Among all three types of
%the particles (neutral, negative and positive), the neutral NPs make
%the strongest interactions.  The large NPs (more than $5$ nm diamter)
%can not be stabilized over A$\beta$40 fibril.  Our study provides a
%good understanding about the functions of the NPs over A$\beta$40
%fibril with different surface charge. It helps experimentalists to
%engineer the NPs which can effectively interact with A$\beta$40 fibril
%and may prevent the growth of the fibril.

\begin{acknowledgement}
The authors thank Dr. Francesco Stellacci for his valuable discussion
about his experimental results. This work was suppoted by NSF grants.
Some of the calculations were performed in UIC Extreme (super
computing cluster). 
\end{acknowledgement}

\begin{suppinfo}
All the plots containing the change of Coulombic, van der Waals
coupling and total interaction strength with the trajectory of the
simulations
\end{suppinfo}

\bibliographystyle{achemso}
\bibliography{fibril}


\end{document}
